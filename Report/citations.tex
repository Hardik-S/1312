% In-text citation examples (ACL natbib)
% \citet{austin1962} argues that performatives do not describe but perform actions.
% Kannada morphology can signal illocutionary force (\citealp{schiffman1979,steever1998}).
% For formal speech act theory, see \citet{searle1989} and \citet{searle1985}.
% For interpretability in NLP, see \citet{ribeiro2016}.
% For Bharatanatyam context, see \citet{gaston2018}.

% Project-specific references (non-bibliographic)
% Based on the provided source text, the document does not include a formal bibliography or "References" section.
% However, the text explicitly references several technical frameworks, administrative guidelines, and internal
% configuration files that are foundational to the project.
%
% 1. Technical Frameworks and Tools
% - MediaPipe Hand Landmarker: primary computer vision tool for the "Live Demo" component and real-time hand tracking.
% - MediaPipe Tasks: implementation method used within the static web app to process hand landmarks.
% - Web Technologies: standard HTML/CSS/JS used for the static web application structure.
%
% 2. Academic and Procedural Guidelines
% - ACL Rolling Review (ARR) Guidelines: requirements tracked in the submission console (anonymity, limitations, formatting).
% - Responsible NLP Checklist: referenced as a requirement and incorporated into metadata and readiness checks.
% - ACL Ethics and Anonymity Policies: referenced for anonymized review, ethics disclosures, and reviewer obligations.
%
% 3. Internal Configuration References
% - mushti-requirements.json: configuration data related to the "Mushti" (fist) detection.
% - movements.json: parameters for motion classification (steadiness vs. courage).
% - requirements-info.json: definitions and text for the ARR/ACL compliance checklists.
%
% 4. Disclosure References
% - AI Assistance Disclosure: requirement to disclose use of generative tools for writing or coding within the Responsible NLP checklist.
%
% Note: The sources provided do not contain specific publication dates, URLs, or formal citation strings for the external
% tools and guidelines listed above.

\begin{thebibliography}{}

\bibitem[Anonymous, n.d.]{anonymous_nd}
Anonymous. n.d.
\newblock Mushti motion classifier submission console.
\newblock ACL Rolling Review Submission.

\bibitem[Chakraborty et~al., n.d.]{chakraborty_nd}
Aishika Chakraborty, Syed Wasif Moin, Arpita Dey, and Ankita Bose. n.d.
\newblock Dance (Bharatanatyam): The art of non verbal communication.
\newblock \textit{International Journal of English Learning \& Teaching Skills}.

\bibitem[Joshi and Jadhav, 2019]{joshijadhav2019}
Manish Joshi and Sangeeta Jadhav. 2019.
\newblock An extensive review of computational dance automation techniques and applications.
\newblock arXiv preprint.

\bibitem[Mallick et~al., 2020]{mallick2020}
Tanwi Mallick, Patha Pratim Das, and Arun Kumar Majumdar. 2020.
\newblock Bharatanatyam dance transcription using multimedia ontology and machine learning.
\newblock arXiv preprint arXiv:2004.11994.

\bibitem[Patel-Grosz et~al., 2018]{patelgrosz2018}
Pritty Patel-Grosz, Patrick Georg Grosz, Tejaswinee Kelkar, and Alexander Refsum Jensenius. 2018.
\newblock Coreference and disjoint reference in the semantics of narrative dance.
\newblock In Uli Sauerland and Stephanie Solt (eds.), \textit{Proceedings of Sinn und Bedeutung 22}, vol. 2, 199--216. Berlin: ZAS.

\bibitem[Rama, 2021]{rama2021}
Siri Rama. 2021.
\newblock Multivector model analysis of hastas or hand gestures as bio semiotic conveyors of rasa.
\newblock \textit{Indica}.

\bibitem[Ryzhakova, 2019]{ryzhakova2019}
Svetlana Ryzhakova. 2019.
\newblock Hasta, mudra, viniyoga. Hand gestures in Indian culture: Problems of origin and sense making.
\newblock \textit{Scientific Articles}.

\bibitem[Sahayaraani, n.d.]{sahayaraani_nd}
K. Sahayaraani. n.d.
\newblock Semiotics of Indian classical dance: A study of Bharatanatyam.
\newblock \textit{International Journal on Science and Technology}.

\bibitem[Tandon, 2020]{tandon2020}
Garima Tandon. 2020.
\newblock The asamyuta hastas of Abhianyadarpana and Natyashastra: (In context of text and performing tradition).
\newblock \textit{Sangeet Galaxy} 9(2): 20--33.

\bibitem[Tecel\~ao, n.d.]{tecelao_nd}
Andr\'e-Luiz Tecel\~ao. n.d.
\newblock The structure of hand gestures in Indian dancing according to Bharata’s \textit{Nāṭya-Śāstra}.
\newblock Scribd.

\end{thebibliography}
